%
% \documentclass[11pt]{article}
%
% \newcommand{\numpy}{{\tt numpy}}    % tt font for numpy
%
% \topmargin -.5in
% \textheight 9in
% \oddsidemargin -.25in
% \evensidemargin -.25in
% \textwidth 7in
%
% \begin{document}
%
% % ========== Edit your name here
% \author{Leones (0,0,0)}
% \title{Number Theory}
% \maketitle
%
% \medskip
%
% ========== Begin answering questions here

\subsection{Identities}
\begin{enumerate}

\item
\begin{equation}
  {n \choose n} = {n \choose 0} = 1
\end{equation}

\item
\begin{equation}
  {n \choose 1} = n
\end{equation}

\item
\begin{equation}
  {n \choose k} = {n \choose n-k}
\end{equation}

\item
\begin{equation}
  {n-1 \choose k-1} = {n-1 \choose k} = {n \choose k}
\end{equation}

\item
\begin{equation}
  2^n = \sum_{i = 0}^{n} {n \choose i}
\end{equation}

\item
Hockey Stick Rule - Pascal triangle
\begin{equation}
  \sum_{i = 0}^{r}{C_{i}^{n+1}} = \sum_{i = 0}^{r}{C_{n}^{n+i}} = {C_{r}^{n+r+1}} = {C_{n+1}^{n+r+1}}
\end{equation}

\item
Burnside's lemma

(useful in taking account of symmetry when counting mathematical objects)
\begin{equation}
  |X/G| = \frac{1}{|G|} \sum_{g\in G} |X^{g}|
\end{equation}


\end{enumerate}

% % \end{document}
% \grid
% \grid
