%
% \documentclass[11pt]{article}
%
% \newcommand{\numpy}{{\tt numpy}}    % tt font for numpy
%
% \topmargin -.5in
% \textheight 9in
% \oddsidemargin -.25in
% \evensidemargin -.25in
% \textwidth 7in
%
% \begin{document}
%
% % ========== Edit your name here
% \author{Leones (0,0,0)}
% \title{Number Theory}
% \maketitle
%
% \medskip
%
% ========== Begin answering questions here



% ========== Begin answering questions hereo
\subsection{Teoremas}
\begin{enumerate}

\item
Teorema de Fermat de la suma de 2 cuadrados:
Un primo puede ser expresado como la suma de 2 cuadrados
\begin{equation}
  p=x^2+y^2
\end{equation}
si: $p\equiv 1 (mod 4)$

\item
Teorema de la suma de 2 cuadrados:
Un numero n puede ser expresado como la suma de 2 cuadrados si y solo si su factorización en primos no contiene ningun $p \equiv 3(mod 4) $ elevado a una potencia impar.

\item
Teorema de Wilson:
\begin{equation}
    p|(p-1)!+1
\end{equation}

\item
Prime number theorem \begin{equation} \phi(n)/n = n/log(n) \end{equation}
Probabilidad de que un numero menor que n sea primo.
\item
El producto de N numeros enteros consecutivos es divisible entre el producto de los primeros N enteros positivos.
\end{enumerate}

\section{Numeros}

\begin{enumerate}
\item
Lucas Numbers:
\begin{equation}
    \sum L_1 = 1
\end{equation}

\begin{equation}
    \sum L_n = F_{n-1}+F_{n+1}
\end{equation}

\item
Twin primes:
\begin{equation}
    T(p_i,p_j) si |p_i-p_j|=2
\end{equation}

\begin{equation}
    a^{\phi(m)}\equiv 1(mod m)
\end{equation}
a y m son coprimos

\item
\begin{equation}
    \phi(ab)=\phi(a)*\phi(b)*\frac{d}{\phi(d)}
\end{equation}
donde d es gcd(a,b)

\end{enumerate}

\subsection{Formulas}
\begin{enumerate}
\item
Si a,b impar:
\begin{equation}
   a^2-b^2 |8
\end{equation}
\item
\begin{equation}
  n/(n-1)/(n-2).../1=\frac{n^2}{n!}
\end{equation}
\item
\begin{equation}
    n(n^2-1)(3n+2) | 24
\end{equation}
\item
\begin{equation}
    \sum i^2=\frac{n(n+1)(2n+1)}{6}
\end{equation}
\item
\begin{equation}
    \sum x^i=\frac{x^n-1}{x-1}
\end{equation}
\item
\begin{equation}
    \sum x^3=(\sum x)^2
\end{equation}
\item
\begin{equation}
    1*2+2*3+3*4+4*5...=\frac{n(n+1)(n+2)}{3}
\end{equation}
\item
\begin{equation}
    \sum F_i = F_(n+2)-1
\end{equation}
Donde F don los numeros de Fibonnaci

\end{enumerate}

\subsection{Ternas pitagoricas}

\begin{enumerate}
\item
Sea gcd(m,n)=1
\begin{equation}
  a=n^2-m^2
\end{equation}
\begin{equation}
  b=2mn
\end{equation}
\begin{equation}
  c=m^2+n^2
\end{equation}

\item
En la terna pitagorica:
\begin{enumerate}
  \item
  Exactamente uno entre a y b, es multiplo de 2.
  \item
  Exactamente uno entre a y b, es multiplo de 3.
  \item
  Exactamente uno entre a y b, es multiplo de 4.
  \item
  Exactamente uno entre a,b,c  es multiplo de 5.
\end{enumerate}


\subsection{LCM Y GCD}

\begin{enumerate}
\item
Suma de GCD
\begin{equation}
    \sum gcd(i,n)=\sum_{d|n}{d\phi(n/d)}
\end{equation}
\item
Suma de LCM
\begin{equation}
    \sum lcm(i,n)=n\frac{(1+\sum_{d|n}\phi(d)d)}{2}
\end{equation}
\end{enumerate}

%
%
\subsection{Phi de Euler}
\begin{enumerate}
\item
\begin{equation}
  \phi(p^{k})=p^k-^{k-1}
\end{equation}

\item
\begin{equation}
  \phi (ab) = \phi(a)*\phi(b) * \frac{d}{\phi(d)}
\end{equation}

\item
\begin{equation}
  a^{\phi m} \equiv 1 mod(m)
\end{equation}

\item
\begin{equation}
  a^{\phi m} \equiv 1 mod(m)
\end{equation}
Con m y a coprimos.

\item
\begin{equation}
  a^{n} \equiv a^{n mod \phi (m)} (mod m)
\end{equation}

\item
\begin{equation}
  x^{n} \equiv x^{\phi (m) +( n mod \phi(m))} (mod m)
\end{equation}
Si x y m no son coprimos. y n mayor o igual log m

\end{enumerate}
\subsection{Diphantine Equations}
$ax+by=c$
Tiene solucion si g=gcd(a,b) divide a c.

Todas las soluciones de la ecuacion se pueden escribir de la forma:

\begin{equation}
  x=x_0 +k(\frac{b}{g})
\end{equation}
\begin{equation}
  y=y_0 - k (\frac{a}{g})
\end{equation}
\end{enumerate}
